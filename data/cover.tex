
%%% Local Variables:
%%% mode: latex
%%% TeX-master: t
%%% End:
%\secretlevel{绝密} \secretyear{2100}

\ctitle{%SUNIST 等离子体参数的原子发射光谱诊断
%SUNIST 球形托卡马克的原子发射光谱诊断研究
SUNIST 等离子体电子温度与\\ 密度的原子发射光谱诊断
}
% 根据自己的情况选,不用这样复杂
%\makeatletter
%\ifthu@bachelor\relax\else
%  \ifthu@doctor
    \cdegree{工学博士}
%  \else
%    \ifthu@master
%      \cdegree{工学硕士}
%    \fi
%  \fi
%\fi
%\makeatother


\cdepartment[工物系]{工程物理系}
\cmajor{核科学与技术}
\cauthor{谢会乔}
\csupervisor{高 喆教授}
% 如果没有副指导老师或者联合指导老师,把下面两行相应的删除即可。
%\cassosupervisor{陈文光教授}
%\ccosupervisor{某某某教授}

% 日期自动生成,如果你要自己写就改这个cdate
% \cdate{\CJKdigits{\the\year}年\CJKnumber{\the\month}月}
\cdate{二〇一四年四月}
% 博士后部分
% \cfirstdiscipline{计算机科学与技术}
% \cseconddiscipline{系统结构}
% \postdoctordate{2009年7月——2011年7月}

\etitle{Optical Emission Spectroscopy of Electron Temperature and Density in SUNIST
%Investigation on the Optical Emission Spectroscopy in the SUNIST Spherical Tokamak
}
% 这块比较复杂,需要分情况讨论:
% 1. 学术型硕士
%    \edegree:必须为Master of Arts或Master of Science(注意大小写)
%              “哲学、文学、历史学、法学、教育学、艺术学门类,公共管理学科
%               填写Master of Arts,其它填写Master of Science”
%    \emajor:“获得一级学科授权的学科填写一级学科名称,其它填写二级学科名称”
% 2. 专业型硕士
%    \edegree:“填写专业学位英文名称全称”
%    \emajor:“工程硕士填写工程领域,其它专业学位不填写此项”
% 3. 学术型博士
%    \edegree:Doctor of Philosophy(注意大小写)
%    \emajor:“获得一级学科授权的学科填写一级学科名称,其它填写二级学科名称”
% 4. 专业型博士
%    \edegree:“填写专业学位英文名称全称”
%    \emajor:不填写此项
\edegree{Doctor of Philosophy}
\emajor{Nuclear Science and Technology}
\eauthor{Xie Huiqiao}
\esupervisor{Professor Gao Zhe}
%\eassosupervisor{Chen Wenguang}
% 这个日期也会自动生成,你要改么?
\edate{April, 2014}

% 定义中英文摘要和关键字
\begin{cabstract}
%
%氦原子的发射光谱强度比诊断%电子温度和密度的手段
%在托卡马克等离子体研究中受到了重视,
%%它首先具有一般发射光谱诊断的优点,如不干扰等离子体、测量设备简单且不易受托卡马克复杂电磁场环境影响等。另外,作为未来聚变等离子体中的固有元素,氦原子具有自旋单重态和三重态两套自旋能级系统,可以在不为等离子体带来其它杂质粒子的前提下,利用来自不同自旋系统的谱线强度比同时确定电子温度和密度。
%%氦原子谱线比法诊断适用的等离子体参数与 SUNIST 等离子体具有相同的范围。且做为小型托卡马克装置,SUNIST 需要建立起简便易用的等离子体参数诊断的手段。
%其适用的电子温度和密度
%%所适用的
%参数范围与如 SUNIST 的小型托卡马克装置或大型装置的边界与偏滤器区等离子体吻合。
%%谱线比法诊断手段的研究和建立不但可以为 SUNIST 建立发射光谱诊断系统,也可以为其他托卡马克装置上的诊断研究提供参考。
%本课题围绕 SUNIST 氦放电等离子体原子发射光谱诊断手段的建立开展。在诊断理论模型方面,针对 SUNIST 氦放电等离子体的特点建立了碰撞辐射模型,重点研究了原子反应速率系数不确定性至激发态粒子数密度计算误差的传递与模型中需包含的激发态能级粒子与反应过程,%根据实际谱线测量情况,
%选择合适的谱线建立了谱线强度比诊断电子温度和密度的方法。实验方面,首先为 SUNIST 建立了光谱测量系统;%完成了系统中单色仪的谱线准确度、分辨率以及相对响应等参数的标定,降低了信号噪声,并消除了信号中存在的基线干扰。
%其次,通过对 SUNIST 放电的进气系统进行重新设计和时序调整,改善了放电的重复性,为基于重复放电的光谱测量奠定了基础;最后,给出了 SUNIST 上光谱诊断测量的结果,通过与微波干涉仪诊断结果的对比验证了谱线比法的可靠性,并%通过多余谱线测量的激发态能级粒子相对数密度对
%复核了碰撞辐射模型的计算结果。%进行了复核。
%研究中还观察到谱线比法与微波干涉仪诊断的电子密度比值与等离子体空间分布状态呈现出一定的函数关系,以及原子谱线强度信号与磁探针信号具有一致的涨落行为等趋势。
%
%本课题研究中开展的创新性工作主要包括:
%\begin{enumerate}
%  \item 明确给出了原子反应速率系数不确定性至激发态粒子数密度计算误差的传递函数。利用此传递函数可以%根据激发态粒子数密度计算精度要求
%      对反应速率系数精度提出具体要求,或在碰撞辐射模型中使用的速率系数精度确定后,估算出激发态粒子数密度的计算误差。这种方法比常规的对速率系数进行扰动并重新求解速率方程的方法简洁直观,且物理意义明确,对碰撞辐射模型的建立及评估具有指导意义。
%  \item 对碰撞辐射模型做出了简化,建立了谱线比法诊断电子温度与密度的手段。通过对比研究发现,在 SUNIST 氦等离子体参数范围下,碰撞辐射模型包含至最高 $n=7$ 壳层能级粒子时即给出可接受的结果。以此为基础,%在 SUNIST 装置上开展了氦放电等离子体原子发射光谱的诊断工作,
%      利用谱线比法诊断了 SUNIST 氦等离子体的电子温度和密度,结果可信。此工作也为具有相同参数范围的其他小型装置或大型装置的边界或偏滤器等离子体的诊断研究提供了参考。
%  \item 为进一步丰富和深入光谱诊断研究提供了方向和思路。%通过对光谱诊断结果的分析,
%      论文观察到如谱线比法与微波干涉仪测量的弦平均电子密度的比例关系与电子密度峰化具有一定的关系、光谱信号与磁探针信号具有一致的涨落行为等趋势。
%\end{enumerate}

光谱诊断是等离子体诊断的主要手段之一,因此对于光谱诊断方法本身的研究也就具有重要的意义。本论文围绕 SUNIST 球形托卡马克装置上光谱诊断的发展,开展了氦放电等离子体原子发射光谱诊断电子温度和密度的研究。在碰撞辐射模型发展上,本论文针对 SUNIST 参数范围的等离子体,对氦原子各能级的主要反应过程及杂质离子可能的影响进行了评估,列出了描述各能级粒子数反应速率的碰撞辐射模型方程;重点研究了原子反应速率系数不确定性至激发态粒子数密度计算误差的传递,从而可以在可接受的误差条件下确定模型中所需包含的激发态能级,在 SUNIST 参数范围下,包含至最高 $n = 7$ 壳层能级粒子时即给出可接受的结果;基于谱线强度比,进而为 SUNIST 建立了电子温度和密度的光谱诊断方法。在诊断系统建立和实验开展方面,通过论文工作,为 SUNIST 建立了光谱诊断系统,对系统进行了标定,实现了基于重复放电的原子发射谱线测量,给出了 SUNIST 上光谱诊断测量的电子温度和密度结果,通过与微波干涉仪等其他诊断结果的对比验证了谱线比法的可靠性。研究中还针对光谱诊断信号中的一些细节,如谱线比法得到的密度与微波干涉仪诊断得到密度的关系、谱线强度信号的涨落等,开展了初步的探索研究。

本文研究中开展的创新性工作主要包括:

1. 明确给出了原子反应速率系数不确定性至激发态粒子数密度计算误差的传递函数。利用此传递函数可以对反应速率系数精度提出具体要求,或在碰撞辐射模型中使用的速率系数精度确定后,估算出激发态粒子数密度的计算误差。这种方法比常规的对速率系数进行扰动并重新求解速率方程的方法简洁直观,且物理意义明确,对碰撞辐射模型的建立及评估具有指导意义。

2. 发展了 SUNIST 氦等离子体参数范围下利用谱线比同时获得电子温度与密度的诊断方法。以此为基础,在 SUNIST 装置上建立起光谱诊断系统,并在实验中给出了可信的诊断结果。此方法也适用于其他装置中具有类似参数范围的等离子体的诊断(如其他包括芯部在内的小型托卡马克装置等离子体或大型装置的边界及偏滤器等离子体等)。

3. 论文观察到如谱线比法与微波干涉仪测量的弦平均电子密度的比例与电子密度峰化具有一定的关系、光谱信号与磁探针信号具有一致的涨落行为等趋势,为进一步丰富和深入光谱诊断研究提供了思路。
\vspace{-0.05em}
\end{cabstract}

\ckeywords{球形托卡马克, 发射光谱诊断, 碰撞辐射模型, 谱线比法}

\begin{eabstract}
The atomic emission spectroscopy is one of the key diagnostics for tokamak plasma research, and, therefore, to investigate the method of spectroscopy diagnostics is of great importance. The dissertation is devoted to develop a spectroscopy diagnostic method for determination of the electron temperature and density in helium plasmas and finally to establish an spectroscopy diagnostics system in the SUNIST spherical tokamak

In the dissertation a collisional-radiative (CR) model is developed for helium plasmas within the parameter ranges of SUNIST helium discharging plasmas. The significance of reaction rates of the number densities of helium excited states is evaluated for collisions with electrons and heavy particles, then the equations of the collision-radiative atomic processes are established. Especially, the propagation of the uncertainties in the reaction rate coefficients of atomic processes is analyzed in details. Through the analysis, the maximum principle quantum number of the excited states in the CR model can be determined according to error in the model. For SUNIST helium plasmas, it is found that the CR model can give acceptable calculations when the maximum principle quantum number of included excited states equals $7$. Finally a line-ratio method is established by selecting appropriate line emissions for the helium discharges of SUNIST.

On the hardware, an atomic emission spectroscopy system is constructed. The line emissions of the plasmas are measured by a shot to shot method based on the repetition of the discharges. Then by employing the CR model we developed, the electron temperature and density are obtained by the method of spectroscopy diagnostics. The diagnostic results are confirmed to be trustable by comparing with those from other diagnostics, such as the microwave interferometry. In additional, some other preliminary results of measured line emission signals are found, such as the ratio of the measured electron density by the line-ratio method to that by the microwave interferometry is closely related with the spatial distribution of the plasma, and the intensities of line emissions have the same time-frequency fluctuation behaviors with those of the signals measured by the magnetic probes.

Some highlighted works in this dissertation include:

1. An error propagation function has been deduced for evaluation of the influences of the uncertainties of rate coefficients of the atomic processes on the calculated number densities of the excited states by the CR model. By using the error propagation function, one can raise a claim on the uncertainties of the atomic reaction rate coefficients directly, or, calculate the error of number densities of the excited states when the rates coefficients are given in the CR model. This method is simple but has a clear physics meaning compared to the traditional method, by which the CR mode is re-solved with perturbed rate coefficients. This error propagation function method developed in the dissertation is expected to play an important role in the building and the evaluation of the CR model.

2. A line-ratio method is established for diagnosing the electron temperature and density of plasma. Based on the method, electron temperature and density are obtained in the helium plasma of SUNIST and the results have been confirmed to be trustable. This line-ratio method can provide reference for the diagnostics of plasmas those have the similar range of plasma parameters with SUNIST, such as core plasma in small tokamaks and edge/divertor plasma in large tokamaks.

3. Some ideas for further research in the atomic emission spectroscopy field are proposed. The ratio of measured electron density by line-ratio method to that by microwave interferometry is closely related with the spatial distribution profile of the plasma. This fact will provide us a method to diagnose the density profile of the plasma. Another experimental observation is that the signals of line emissions have the same fluctuation behaviors with those measured by the magnetic probes, and it will provide us a possibility of investigating the MHD behaviors in plasmas by optical spectroscopy diagnostics.
\end{eabstract}

\ekeywords{spherical tokamak, atomic emission spectroscopy, collisional-radiative model, line ratio method}
